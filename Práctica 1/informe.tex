\documentclass[11pt,leqno]{article}
\usepackage[spanish,activeacute]{babel}
\usepackage[utf8]{inputenc}
\usepackage{amsfonts}
\usepackage{enumerate}
\usepackage{listings}
\usepackage{amsthm}
\usepackage[hidelinks]{hyperref}
\usepackage{booktabs}
\usepackage{siunitx}
\usepackage{pgfplotstable}
\usepackage{verbatim}
\usepackage{geometry}
\usepackage{changepage}

\author{Jacinto Carrasco Castillo 	\\
		N.I.F. 32056356-Z			\\ 
		\href{jacintocc@correo.ugr.es}{jacintocc@correo.ugr.es}}
		
\title{	Práctica 1 Metaheurísticas.\\
		Búsqueda por trayectorias para el problema \\
		de la selección de características\\
		Curso 15-16\\
		Algoritmos: SFS, LS, SA, TS, Extended TS}

\newcommand{\maketable}[1]{
\begin{adjustwidth}{-1cm}{}
\resizebox{\linewidth}{!}{
\pgfplotstabletypeset[
	every head row/.style={
		before row={%
				\hline
				& \multicolumn{4}{c|}{WDBC} & \multicolumn{4}{c|}{Movement Libras} & \multicolumn{4}{c|}{Arrythmia}\\
				\hline
		},
		after row=\hline\hline
	},
	column type/.add={}{|},
	every last row/.style={before row =\hline\hline, after row=\hline},
	every first column/.style={ column type/.add={|}{} },
	columns/partition/.style={column name = , string type},
	columns/in/.style={column name =\%Clas. in},
	columns/inL/.style={column name =\%Clas. in},	
	columns/inA/.style={column name =\%Clas. in},
	columns/out/.style={column name =\%Clas. out},
	columns/outL/.style={column name =\%Clas. out},	
	columns/outA/.style={column name =\%Clas. out},
	columns/T/.style={column name =T},
	columns/tA/.style={column name =T},
	columns/tL/.style={column name =T},
	columns/red./.style={column name =\%red.},
	columns/redL/.style={column name =\%red.},	
	columns/redA/.style={column name =\%red.},
	precision=4
	]{#1}
}
\end{adjustwidth}
}

\newcommand{\makeresume}[4]{
\pgfplotstablevertcat{#1}{#2}
\pgfplotstablecreatecol[copy column from table={#3}{in}]{inL} {#1}
\pgfplotstablecreatecol[copy column from table={#3}{out}]{outL} {#1}
\pgfplotstablecreatecol[copy column from table={#3}{red}]{redL} {#1}
\pgfplotstablecreatecol[copy column from table={#3}{T}]{tL} {#1}
\pgfplotstablecreatecol[copy column from table={#4}{in}]{inA} {#1}
\pgfplotstablecreatecol[copy column from table={#4}{out}]{outA} {#1}
\pgfplotstablecreatecol[copy column from table={#4}{red}]{redA} {#1}
\pgfplotstablecreatecol[copy column from table={#4}{T}]{tA} {#1}
}


\begin{document}

% Tablas KNN
\pgfplotstableread[col sep=comma]{Resultados/wKNN.csv}\datawKNN
\pgfplotstableread[col sep=comma]{Resultados/lKNN.csv}\datalKNN
\pgfplotstableread[col sep=comma]{Resultados/aKNN.csv}\dataaKNN

% Tablas Greedy SFS
\pgfplotstableread[col sep=comma]{Resultados/wSFS.csv}\datawSFS
\pgfplotstableread[col sep=comma]{Resultados/lSFS.csv}\datalSFS
\pgfplotstableread[col sep=comma]{Resultados/aSFS.csv}\dataaSFS

% Tablas Local Search
\pgfplotstableread[col sep=comma]{Resultados/wLS.csv}\datawLS
\pgfplotstableread[col sep=comma]{Resultados/lLS.csv}\datalLS
\pgfplotstableread[col sep=comma]{Resultados/aLS.csv}\dataaLS

\begin{comment}
% Tablas Simulated Annealing
\pgfplotstableread[col sep=comma]{Resultados/wSA.csv}\datawSA
\pgfplotstableread[col sep=comma]{Resultados/lSA.csv}\datalSA
\pgfplotstableread[col sep=comma]{Resultados/aSA.csv}\dataaSA

% Tablas Tabu Search
\pgfplotstableread[col sep=comma]{Resultados/wTS.csv}\datawTS
\pgfplotstableread[col sep=comma]{Resultados/lTS.csv}\datalTS
\pgfplotstableread[col sep=comma]{Resultados/aTS.csv}\dataaTS

% Tablas Extended Tabu Search
\pgfplotstableread[col sep=comma]{Resultados/wETS.csv}\datawETS
\pgfplotstableread[col sep=comma]{Resultados/lETS.csv}\datalETS
\pgfplotstableread[col sep=comma]{Resultados/aETS.csv}\dataaETS

\end{comment}

\begin{titlepage}
\maketitle
\end{titlepage}
\tableofcontents
\newpage

\section{Descripción del problema}

\section{Descripción de la aplicación de los algoritmos}

\section{Estructura del método de búsqueda}

\section{Algoritmo de comparación}

\section{Procedimiento para desarrollar la práctica}

\section{Experimentos y análisis de resultados}

% Tablas Resumen KNN
\makeresume{\dataKNN}{\datawKNN}{\datalKNN}{\dataaKNN}

% Salida por pantalla
\maketable{\dataKNN}

% Tablas Resumen SFS
\makeresume{\dataSFS}{\datawSFS}{\datalSFS}{\dataaSFS}

% Salida por pantalla
\maketable{\dataSFS}

% Tablas Resumen LS
\makeresume{\dataLS}{\datawLS}{\datalLS}{\dataaLS}

% Salida por pantalla
\maketable{\dataLS}

\section{Bibliografía}

\end{document}
